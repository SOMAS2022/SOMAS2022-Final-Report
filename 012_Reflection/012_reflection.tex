\chapter{Reflection}\label{sec:reflection}

Ultimately, this project was interesting, challenging, and very teaching, and there is a great deal we as a class can learn from this experience. When beginning this undertaking, many of us were very involved and excited about how to best self-organize and accomplish this task. However, there were many who took much longer to organize into teams, voice their opinions, and show that they understood the facets of the course, leading to delays in the process of building out our game. In retrospect, there seem to have been misunderstandings regarding what exactly was expected of us. An example of this is our proposal of utilizing a force fight imposition and forgoing disobedience, rather than focusing on institutionalized power and sanctions in response to disobedience. As such, we should have set clearly defined goals regarding what we wanted to achieve in designing this game, and how each facet of this related to what we learned in the course. Lacking the early depth of understanding required to successfully implement Ostom intitutions with social choice theory cost time and resources, as the group course-corrected for every facet missed. 

One clear improvement we could have made, is better communication. We can categorize communication into two main types: communication of ideas that we want to see in the game, and communication of what has been implemented for the game. Regarding ideas, many of us seemed to have focused directly on the game during the early stages rather than the course. This led to individuals and teams realizing much later that their agent had not incorporated the core elements or principals of the course content; however, this put a great degree of stress on the infrastructure team as it meant changing much of what had already been agreed upon. A clear solution to this is ensuring that members of each team understand what they are trying to accomplish from the beginning and making sure that they voice these ideas to the infrastructure and game design teams. One recommendation could be to implement some method of holding individuals accountable for their learning of the course material. This could be in the form of in-class quizzes, class-participation marks, or some other mandatory class involvement. The main logic behind this recommendation is because it seems many individuals from the start were more focused on their mark rather than the course material. As such, a method of ensuring a widespread understanding of course material from early on might allow more people to be better equipped to voice their opinions throughout the course rather than just at the end. Regarding the communication of implementation, many teams seemed to have a limited understanding of what exactly had been implemented and how they utilize these implementations. This was because of poor documentation, and subpar discussion between the infrastructure, game design, and individual agent teams. One improvement could be to ensure there are clear leaders from each team that give consistent updates regarding what has been designed or implemented and how people can use it when coming up with their own agents.  

Some other issues were present regarding the implementation of the infrastructure. As the messaging system did not include a number of the desire subject matters, teams were prevented from implementing agents that could get a strong idea of other agent's performance, and their reputations. This in turn impacted the amount of information that could be sent to the chair which ultimately prevented governance implementation from being as sophisticated as some teams planned. This seems to tie back to the issue of communication mentioned earlier. In retrospect, defining clear goals of what we wanted for the project, and why we wanted these goals achieved would be much more conducive to a project that emulates the facets of self organising multi-agent systems.