\chapter{Introduction}\label{introduction}

\section{Introduction}\label{sec:intro}

% short introdcution to the module and the objectives/goals of the SOMAS cohort. 
% Some background on Self-Organising Multi-Agent System and its facets
% Ostrom Principals 
% Institutionalised power
% Game theory and strategic interaction 
% Social Choice Theory 
% alternative Dispute Resolution 
% Computational Justice
% Knowledge Agreggation 
% Sanctions and Dissent 


\subsection{Story Line}\label{sec:story line}

% story line
For the implemented game, the following story-line was created: A rag-tag group of rebel peasants find themsevles at the bottom of a deep and dark pit. To escape to freedom, the peasants must battle through each rising level of the pit, slaying a number of perilous beasts on their way. With each enemy slayed at the blood soaked hands of the proletarian, they gain access to new and improved weaponry to fight on and never surrender. Once free, the peasants can continue their struggle against the totalitarian dictatorships of the western world, until, in God's good time, the New World, with all its power and might, steps forth to the rescue and the liberation of the old.  

To achieve all these goals, peasants must self-organise to survive. This can be done by electing governors, conducting collective risk analysis, assigning common pool resources and performing alternative dispute resolution using institutionalised power. All of these aspects, revolve around the understanding of social choice theory and knowledge agreggation. 


\subsection{Rules, Attributes \& Artifacts}\label{sec:rules}

The objective of the game is for an Agent to win, individually, by escaping the pit with as many `hit-points' and `health points' as possible, whilst also ensuring that enough of your fellow players survive to start the revolution upon esape. 
For each level within the pit, combat consists of a number of fight rounds which continue until either; the enemy is defeated, or the agents turned into chutney. Within each round, Agents can either; attack, defend, or cower. The first action deals damage to the enemy, the second absorbs damage from the enemy, and the final option skips the round whilst regenerating health and stamina.
Agents will be granted multiple opportunities within fight rounds and pit levels to self-organise. These self-organisational tasks could involve the allocation of loot dropped by a vanquished foe. This perticular example occurs at the end of each fight, before advancing to the next level of the pit. Expansion and explination of each of the task will be introduced later in this report. 

Before listing the rules of the games, it is important to give a simple overview of the Agent and enemy attributes as well as the equipment that can be used within the game.  

\subsubsection{Attributes \& Equipment}

All Agents within the game have the following attributes:

\begin{itemize} 
    \item Health Points ($HP$)
    \item Stamina Points ($ST$)
    \item Attack Base ($A_s$)
    \item Attack Bonus ($A_b$)
    \item Defense Base ($D_s$)
    \item Defense Bonus ($D_b$)
\end{itemize}

Attack and defense bonuses represent the value of the equipment that is currently being used by the agent. The base values are native to all agents and remain contant, the minimum fight potential of an agent. For the enemy's attributes:

\begin{itemize}
    \item Resilience ($X$)
    \item Damage ($Y$)
\end{itemize}

Equipment is divided into two sections, weapons and potions. Each weapon is of a specific hit value and can be equiped by an Agent to increase their damage (or shield) potential. There are only two weapon choices:

\begin{itemize}
    \item Sword ($A_b$)
    \item Shield ($D_b$)
\end{itemize}

Swords are used by Agents who attack, and shields are used by Agents who defend. In both cases, their potential would be the conbination of both base and bonus attribute values, giving the Total Attack (/Defend). Potions are consumed by Agents to regenerate their attributes, specifically $HP$ and $ST$. Again, there is a specific value assigned to each new potion. 

\begin{itemize}
    \item Health Potion ($P_{hp}$)
    \item Stamina Potion ($P_{st}$)
\end{itemize}

The value mechanisms for all these variables will be outlined in Section. ADD\_MATHS\_SECTION\_NUMBER. 

\subsubsection{Rules}
Now that the basic game variables have been defined, the standard rules of the game can be listed. 

\begin{enumerate}
    \item All agents start with equal $ST$, $HP$
    \item An enemy dies if $\sum_{i} (A_b_i + A_s_i) > X_{remaining}$ at the end of a round
    \item Damage dealt to agents who fight = $\gamma Y - \sum_{i} (D_b_i + D_s_i)$ 
    \item An agent dies if $\frac{\gamma Y - \sum_{i} (D_b_i + D_s_i)}{N} > HP$
    \item To attack: $ST \geq (A_b + A_s)$
    \item To defend: $ST \geq (D_b + D_s)$
    \item If an agent attacks, $ST -= (A_b + A_s)$
    \item If an agent defends, $ST -= (D_b + D_s)$
    \item If an agent cowers, $HP += HP_c$, $ST += ST_c$
    \item If all agents cower, all recieve damage $\gamma Y$
    \item If, at the end of a round, $N<M$, the game ends 

    \item Agent can hold extra equipment in their inventory
    \item Agents can donate $HP$ to the $HP_{pool}$ at the end of each level
    \item If $HP_{pool} > X$, agents pass level without fighting

    \item Before each round, agents may communicate to decide fight actions and Chairs may issue sanctions to those to defected from the group strategy (if they wish)
    \item At the end of each level, agents may communicate to allocate the common resource pool, elect new governments, trade equipment and resolve any disputes.


\end{enumerate}

% Agent / Enemy Attributes and variables 
% specifications and Rules
% standard rules

Besides these standard rules, there were a number of additions that were made. To add new dimentions to the game's social ascept, agents only have access to granulated views of their fellow agents $HP$ and $ST$ attributes. This addition creates a high degree of uncertainty in agent calcualtions and inferences which could create interesting outcomes or large changes in group fight decisions. An agents inventory will be private, allowing agents to hoard equipment without others knowing. This too allows a higher range of agent strategies and introductes more oportunities for selfishness and exploitation. 

Fight decision strategies are proposed for the whole group and chosen once a majority has been achieved. Deviation from this aggreed strategy can result in sanctions, determined by the Chair. Sanctions can include being forced to fight for a set number of rounds with zero fight attributes. Potential chairs must run for office on a number of policies and are chosen by plurality during governance elections. Finally, an disputes are resolved by the Chairs individual ideals or by the games determination of group utility. All of these decisions attempt to add the concepts of institutionalised power and social choice theory to the game. The specific mechanisms of collective actions and governance will be further detailed in Section. GAME\_DESIGN.

% more nuanced rules and the reasons for these rules 


\subsection{Team Structure}\label{sec:team struct}

%% input the strucutre of each team
% how does the flow of each team work 
%% input each team and who is in each team 

\subsection{Report Structure}\label{sec:report struct}

% the overall structure of the report
% 
